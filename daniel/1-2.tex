\documentclass{book}

\input{./preamble}

\begin{document}

\color{blue}

The \emph{recursion principle} is the rule that states that
${\rec{A \times B} (f) : A \times B \to C}$ is well defined,
and it is taken as a primitive notion.
We are asked to instead take the well-definedness of ${\proj{1}}$ and
${\proj{2}}$ as primitive instead and to derive the recursion principle.

Definitions of ${\rec{A \times B}}$, ${\proj{1}}$, and ${\proj{2}}$ are
given below for convenience.

\begin{align*}
  &\rec{A \times B} : (A \to B \to C) \to A \times B \to C \\
  &\rec{A \times B} (f) ((a,b)) := f (a) (b) \\
  \\
  &\proj{1} : A \times B \to A \\
  &\proj{1} := \rec{A \times B} (\lambda \, a \, b \mapsto a) \\
  \\
  &\proj{2} : A \times B \to B \\
  &\proj{2} := \rec{A \times B} (\lambda \, a \, b \mapsto b) \\
\end{align*}

(In agreement with the notation used in the text,
${\lambda \, a \, b \mapsto \Phi}$ is merely shorthand for
${\lambda \, a \mapsto (\lambda \, b \mapsto \Phi)}$. We can think of
such an expression as a ``two-variable'' curried function [mmmm,
curry].)

We need primitive definitions of ${\proj{1}}$ and ${\proj{2}}$.

\begin{align*}
  &\proj{1} : A \times B \to A \\
  &\proj{1} (a,b) := a \\
  \\
  &\proj{2} : A \times B \to B \\
  &\proj{2} (a,b) := b \\
\end{align*}

We define

\begin{align*}
  &\rec{A \times B}' : (A \to B \to C) \to A \times B \to C \\
  &\rec{A \times B}' (f) ((a,b)) := f (\proj{1} (a,b)) (\proj{2} (a,b)) \\
\end{align*}

We have

\begin{align*}
  \rec{A \times B} (f) (a,b)
  &= f (a) (b) \\
  &= f (\proj{1} (a,b)) (\proj{2} (a,b)) \\
  &= \rec{A \times B}' (f) (a,b) \\
\end{align*}

So ${\rec{A \times B} = \rec{A \times B}'}$.

\end{document}
